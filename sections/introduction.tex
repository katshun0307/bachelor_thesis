\subsection{背景}

多段階ラムダ計算 \cite{Davies1996a} では, プログラムコードを値として扱い, 
コードの埋め込みや評価を自在に行うことで, 効率の良いコードの生成, 実行を可能にする.

また, 一般に型付きラムダ計算は, カリーハワード同型対応によって自然演繹の直観主義論理体系に対応する.
具体的にはラムダ計算の型が命題に, 項が証明に対応している. Löf は2つの項が同値であるという命題にラムダ計算で対応する型として, 
\emph{同値型}\cite{Lof1984}を提案した.
同値型は, 2つの項を引数としてとる依存型の一種であり, 同値型の項(=証明)が存在すれば, 二つの項は等しい事を表している.

\subsection{目的}

本研究では, 多段階ラムダ計算に依存型と, 依存型を用いた同値型 を追加することで, 
プログラムコードを含む多段階計算上の同値を形式的に定義し, 二つの項が同値であることを, 同値型の項を用いて表現できるようにした.
