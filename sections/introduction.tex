\subsection{背景}


\subsubsection{多段階計算 \lamcirc}
Davies による多段階ラムダ計算の体系 \lamcirc \cite{Davies1996a} では, プログラムコードを値として扱い, 
コードの埋め込みや評価を自在に行うことで, 効率の良いコードの生成, 実行を可能にしている. 

\mynext{t} は, 項 $t$ をプログラムコードとして扱うものであり, \myprev{t} は, 
プログラムコード $t$ の中身を通常の項として取り出すものである.

この仕組みを利用した特徴的な表現として, コードの埋め込みが存在する. コードの埋め込みは, \tnext で表されたコードの内部に,
別のコードに \tprev を適用して埋め込むことができるものである.

例として, \Cref{embed_ex0} は, $\mynext{3 + 3}$ が内部に埋め込まれる事で, \Cref{embed_ex1} のように簡約される.

\begin{equation}
    \mynext{\myprev{\mynext{3 + 3}} + 4}
    \label{embed_ex0}
\end{equation}

\begin{equation}
    \mynext{3 + 3 + 4}
    \label{embed_ex1}
\end{equation}

また, プログラムコードの中身では, 通常の簡約を許さないため, これが $\mynext{10}$ のように評価されることはない. 


\subsubsection{依存型を持つ体系 \lamlf}
依存型は, 内部に項を持つことができるような型の種類である.
依存型を持つ体系として, \lamlf \cite{Harper1993} がある.
\Cref{fig_lf_syntax} に, \lamlf の文法の一部を紹介する.
\lamlf において特徴的なのは, カインドという, 依存型ががどのような型を受け取るかを
示すための文法要素である.
カインドには, 具体型(Proper type) と依存関数型(Dependent product type)の2種類がある.
具体型を持つ型は, 単純型付きラムダ計算のように, 項に型付けされる型である.
対して, 依存関数型を持つ型は, そのカインドで指定された方を持つ項を受け取り,
依存型を返す.
依存関数型は, 型に対するラムダ抽象のようなものと考える事ができる.
% カインドには, 型が実際にその型を持つ項を有するかを表す具体型と, 
% 型がある型を持つ項を受け取り, 別のカインドを持つ型になることを示す dependent product type がある.
% つまり, 項に対する関数型が, 型に対するカインドである.


例えば, ベクトルの長さを表す自然数を内部に保持する型\textrm{Vector} を考える. 
\textrm{Vector} は, ベクトルの長さを表す自然数型 (\textrm{Nat}) の項を受け取り,
その長さのベクトルを表す具体型を返す. この型族(Type family) \textrm{Vector}のカインド判断は,
\lamlf では, 以下のように表される.

$$\Gamma \myvdash{n} \textrm{Vector} \DCOL \Pi x \COL \textrm{Nat}. *$$

更に, \textrm{Vector} が $3$ に適用された場合, $\textrm{Vector} \mysp 3$ は, 具体型となり, 
このカインド判断はこのように記述される.

$$\Gamma \myvdash{n} \textrm{Vector} \mysp 3 \DCOL *$$

\FB
\begin{fig}{\lamlf の文法}
    \begin{align*}
        \textrm{項} && t & ::=
        x \pipe \mylam{x}{T}{t} \pipe
        t \mysp t\\
        \textrm{型} && T & ::=&\\
        && &\pipe X \quad &(\text{型族})\\ 
        && &\pipe \Pi x \COL T.T \quad &(\text{型抽象})\\ 
        && &\pipe T\mysp t \quad &(\text{依存型適用})\\
        \textrm{カインド} && K & ::= &\\
        && &\pipe * \quad &(\text{具体型})\\
        && &\pipe \Pi x \COL T. K \quad &(\text{依存関数型})\\
        \textrm{型判断} && {} & \Gamma \vdash t \COL T\\
        \textrm{カインド判断} && {} & \Gamma \vdash T \DCOL K
    \end{align*}
    \label{fig_lf_syntax}
\end{fig}
\FB

\subsubsection{同値型}

Martin-Löf による同値型\cite{Lof1984}は, 項の間の同値を型として表現できる依存型の一種である.
同値型は, 任意の型$T$に対して, 2つの型$T$を持つ項を受け取る.
以下に同値型に関する \lamlf でのカインド規則を示す.

\begin{figure}[htbp]
    \centering
    \infrule{
        \Gamma \vdash T \DCOL *
    }{
        \Gamma \vdash \mysingleeq{T}\DCOL \Pi x\COL T. \Pi y\COL T. *
    }
\end{figure}
\FB

同値型によって表される同値は, 体系における項の同値によるものであれば表現することができ,
逆に同値でない項同士は, 同値型で表現することはできない.
このような性質から, 


\subsection{目的}

本研究では, 多段階計算の型システムを強力にするため,
上で述べた依存型と同値型に関して拡張する.
また, 拡張した体系において, プログラムコードを含む項の同値性を形式的に定義した.

\subsection{本報告書の構成}

本報告書では, \Cref{sec_system} で 提案言語の文法と型システムを紹介し,
簡約と同値性の定義について記述する. 次に \Cref{sec_property} で体系の性質を示し,
最後に, \Cref{sec_algorithmic} でアルゴリズム的型付けを導入し, 通常の型付け規則に対して健全かつ完全であることを示す.
一部の補題とその証明は, 付録に記載してある.

% next_section: ./type_system.tex
