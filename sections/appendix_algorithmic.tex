
\vspace{10pt}

\begin{jdefinition}[項のレヴェル]\label{DefLevelTerms}
    自然数$n$に対して, 項のレヴェルを以下のように定める.
    \begin{itemize}
        \item レヴェル$0$の項は, 型の内部に存在しない項である.
        \item レヴェル$n$の項の中の型の中に含まれる項は, レヴェル$n + 1$ の項である.
    \end{itemize}
\end{jdefinition}

\begin{jdefinition}[アルゴリズム簡約]\label{DefAlgorithmicReduction}
    インデックス $m \in \mathbb{N}$ を持つアルゴリズム簡約 $\alarrow{m}$ を,
    $m$ インデックスを持つアルゴリズム簡約規則の, 任意のレヴェルの項への適用とする.  
\end{jdefinition}

\begin{jdefinition}[アルゴリズム正規形]\label{DefANF}
    項 $t$ の インデックス$m$を持つ \emph{アルゴリズム的正規形} を,
    インデックス$m$ のアルゴリズム簡約による正規形とし, \anf{m}{t} と表記する.
\end{jdefinition}

\begin{jlemma}[レヴェル0のアルゴリズム簡約の強正規化性]\label{AlgorithmicSN0}
    任意の $m$に対して, インデックス $m$ のレヴェル$0$ の項の簡約は, 強正規化性を持つ.
    \begin{proof}
        \Cref{SN} より明らか.
    \end{proof}
\end{jlemma}

\begin{jdefinition}[\lamlfcirc から \stlc への変換]\label{def_translation_sn}
    \lamlfcirc のカインドと型から単純型付きラムダ計算\stlc の型への変換 \mdflat と, \lamlfcirc の型と項から, \stlc の項への変換 \mdsharp を, 以下のように定義する. 
    ただし, 型$\omega$ は, 項$\rho$, 型 \Unit は 項 \unit を持つシングルトン型である.
    また, \stlc の 任意の型$\sigma$ に対して, 型$\omega \rarrow (\sigma \rarrow \omega) \rarrow \omega$ を持つ項 $\pi_{\btrans{\sigma}}$ を考える.
    \begin{align*}
        &\btrans{\Pi x \COL T. K} &&= \btrans{T} \rarrow \btrans{K}\\
        &\btrans{*} &&= \omega\\
        &\btrans{X} &&= \tilde{X'}\\
        &\btrans{\Pi x \COL T_1. T_2} &&= \btrans{T_1} \rarrow \btrans{T_2}\\
        &\btrans{T \mysp t} &&= \btrans{T}\\
        &\btrans{\textbf{Eq}_{T}} &&= \Unit \rarrow \btrans{T}\\
        &\btrans{\bigcirc T} &&= \Unit \rarrow \btrans{T}\\
        % =============
        &\atrans{X} &&= \tilde{X}\\
        &\atrans{\Pi x \COL T_1 .T_2} &&= \pi_{\btrans{T_1}} \mysp \atrans{T_1} \mysp (\mylam{x}{\btrans{T_1}}{\atrans{T_2}})\\
        &\atrans{T \mysp t} &&= \atrans{T} \atrans{t}\\
        &\atrans{\textrm{Eq}_{T}} &&= \mylam{x}{\btrans{T}}{\mylam{y}{\btrans{T}}{\rho}}\\
        % &\atrans{\bigcirc T} &&= (\mylam{x}{\Unit}{\atrans{T}}) \mysp \unit\\
        &\atrans{\bigcirc T} &&= (\mylam{x}{\btrans{T}}{\rho}) \mysp \atrans{T}\\
        &\atrans{x} &&= \tilde{x}\\
        &\atrans{\mylam{x}{T}{t}} &&= (\lambda y \COL \btrans{T}. \lambda x \COL \omega. \atrans{t}) \atrans{T}\\
        &\atrans{t_1 \mysp t_2} &&= \atrans{t_1} \atrans{t_2}\\
        &\atrans{\myprev{t}} &&= \atrans{t} \mysp \unit\\
        &\atrans{\mynext{t}} &&= \lambda x \COL \Unit. \atrans{t}\\
        &\atrans{\myid{t}} &&= \mylam{x}{\Unit}{\atrans{t}}\\
        &\atrans{\myidpeel{t_{eq}}{x}{t}} &&= (\mylam{a}{\btrans{T}}{[x \mapsto a] \atrans{t}}) \mysp (\atrans{t_{eq}} \mysp \unit) \quad ( t_{eq} \COL \myeq{T}{a}{b} \text{とする})\\
        &\natural(\Gamma ) &&= \{\atrans{x} \COL \btrans{T} \pipe (x \COL T) \in \Gamma \} \cup \{\atrans{X} \COL \btrans{K} \pipe (X \COL K) \in \Gamma\}
    \end{align*}
\end{jdefinition}

\begin{jlemma}[変換と代入]\label{al_trans_and_substitution}
    任意の $x$ と $t$ について, 
    $\btrans{T} = \btrans{[x \mapsto t]T}$ かつ, $\btrans{K} = \btrans{[x \mapsto t] T}$ である.
    \begin{proof}
        \mdflat 変換は, 型に含まれる項を消去していることから, 明らか.
    \end{proof}
\end{jlemma}

\begin{jlemma}[\mdflat 変換と型の同値]\label{al_trans_and_equivalence}
    \mysp
    \begin{itemize}
        \item $\Gamma \myvdash{n} T_1 \myequiv{0} T_2$ であるなら, $\btrans{T_1} = \btrans{T_2}$.
        \item $\Gamma \myvdash{n} K_1 \myequiv{0} K_2$ であるなら, $\btrans{K_1} = \btrans{K_2}$.
    \end{itemize}
    \begin{proof}
        \textbf{導出に関する構造帰納法によって容易に示される.}
    \end{proof}
\end{jlemma}

\begin{jlemma}[変換の型付け関係の保存]\label{algorithmic_trans_type_preserve}
    \mysp
    \begin{itemize}
        \item \lamlfcirc で, $\Gamma \myvdash{n} t \COL T$ ならば, \stlc で, $\natural(\Gamma) \vdash \atrans{t} \COL \btrans{T}$ である.
        \item \lamlfcirc で, $\Gamma \myvdash{n} T \COL K$ ならば, \stlc で, $\natural(\Gamma) \vdash \atrans{T} \COL \btrans{K}$ である.
    \end{itemize}
    \begin{proof}
        \textbf{導出規則に関する帰納法による.}
        \Cref{al_trans_and_substitution,al_trans_and_equivalence} を用いる.
    \end{proof}
\end{jlemma}

\begin{jlemma}[変換による\mbeta 簡約と同値型の項に関する簡約ステップの保存]\label{algorithm_trans_reduction_step}\label{al_beta_eq_trans_reduction_preserve}
    \lamlfcirc で, \mbeta 簡約あるいは, 同値型の項に関する簡約によって, $t_1 \rarrow t_2$ ならば, \stlc で, $\atrans{t_1} \xrightarrow{+} \atrans{t_2}$ が成立する.
    \begin{proof}
        変換の定義より, 示される.
    \end{proof}
\end{jlemma}

\begin{jtheorem}[アルゴリズム簡約の強正規化性]\label{AlgorithmicSN}
    任意の $m$に対して, レヴェル$n$ まで持つ項のインデックス$m$ でのアルゴリズム簡約は, 強正規化性を持つ.
    \begin{proof}
        \textbf{背理法による.}
        \lamlfcirc が強正規化性を持たないと仮定する.
        このとき, \lamlfcirc で型付けされた項$t$が存在し, $t \rightarrow t_1 \rightarrow t_2 \rightarrow \cdots $
        のような無限長の簡約列が存在する. 
        \Cref{FiniteStageReduction}より, この簡約列には無限個の \mbeta-簡約 あるいは 同値型に関する簡約ステップが存在するから,
        \Cref{al_beta_eq_trans_reduction_preserve} より, \stlc{} でも $\strans{t}$ の無限長の簡約列が存在する.
        これは, \stlc が強正規化性を持つことに矛盾する.
    \end{proof}
\end{jtheorem}

\begin{jtheorem}[アルゴリズム簡約の合流性]\label{AlgorithmicConfluence}
    $0$-インデックスのアルゴリズム簡約は, 全てのレヴェルの項について合流性を持つ.
    % The algorithmic reduction with index 0 is confluent for reduction on all levels.
    レヴェル $i$ に関する項の簡約で, $q \iarrow{0} r$ かつ, レヴェル$j$に関する項の簡約で $q \iarrow{0} s$ ならば,
    $r \iarrow{0} t$ かつ $s \iarrow{0} t$ であるような項 $t$ が存在する.
    \begin{proof}
        \begin{itemize}
            \item{$i = j$}\\
            \Cref{ChurchRosser} より, $t$ は存在する.
            \item{$i \neq j$}\\
            あるレヴェルの項での簡約は, 他のレヴェルの項について簡約基を生成しない.
            よって, 2つの簡約の順番の入れ替えによっても, 簡約結果は同じになる.
        \end{itemize}
    \end{proof}
\end{jtheorem}

\begin{jtheorem}[アルゴリズム簡約の型保存]\label{AlgorithmicTypePreservation}
    $\Gamma \myvdash{n} t \COL T$ かつ $t \alarrow{m} t'$ であるなら, $\Gamma \myvdash{n} t' \COL T$.
    \begin{proof}
        \textbf{レヴェル$n$に関する完全帰納法による.}.
        レヴェル0については, \Cref{Preservation} で示されている.
        レヴェル$n$ までの項について示されているなら,
        \fauxsc{QT-App} と \fauxsc{T-Conv} を用いて,
        レヴェル$n+1$ の項についても示すことができる
    \end{proof}
\end{jtheorem}

\begin{jlemma}[アルゴリズム簡約と項の同値]\label{AlgorithmicTermEquivalence}
    $\Gamma \myvdash{n} M \COL T$ かつ $M \alarrow{m} M'$ であるなら, $\Gamma \myvdash{n} M \myequiv{m} M'$である.
    \begin{proof}
        \textbf{$M \alarrow{m}  M'$ の導出の関する帰納法による.}
    \end{proof}
\end{jlemma}

\begin{jtheorem}[アルゴリズム正規形と項の同値]\label{AlgorithmicANFEquivalence}
    $\Gamma \myvdash{n} t \COL T$ ならば, $\Gamma \myvdash{n} t \myequiv{m} \anf{m}{t} \COL T$ である.
    \begin{proof}
        \Cref{AlgorithmicTermEquivalence} と \Cref{AlgorithmicTypePreservation} から示される.
    \end{proof}
\end{jtheorem}

\begin{jlemma}[アルゴリズム同値の反射律]\label{AlRefl}
    任意の $n \in \mathbb{Z}, m \in \mathbb{N}$, に対して,
    \begin{itemize}
        \item $\Gamma \myalvdash{n} K \myequiv{m} K$.
        \item $\Gamma \myalvdash{n} T \myequiv{m} T$.
        \item $\Gamma \myalvdash{n} t \myequiv{m} t$.
    \end{itemize}
    \begin{proof}
        \textbf{導出に関する構造帰納法による.}
    \end{proof}
\end{jlemma}

\begin{jlemma}[アルゴリズム同値の対称律]\label{AlSym}
    任意の $n \in \mathbb{Z}, m \in \mathbb{N}$ に対して,
    \begin{itemize}
        \item $\Gamma \myalvdash{n} K_1 \myequiv{m} K_2$ であるなら, $\Gamma \myalvdash{n} K_2 \myequiv{m} K_1$.
        \item $\Gamma \myalvdash{n} T_1 \myequiv{m} T_2$ であるなら, $\Gamma \myalvdash{n} T_2 \myequiv{m} T_1$.
        \item $\Gamma \myalvdash{n} t_1 \myequiv{m} t_2$ であるなら, $\Gamma \myalvdash{n} t_2 \myequiv{m} t_1$.
    \end{itemize}
    \begin{proof}
        \textbf{導出に関する構造帰納法による.}
    \end{proof}
\end{jlemma}

\begin{jlemma}[アルゴリズム同値の推移律]\label{AlTrans}
    任意の $n \in \mathbb{Z}, m \in \mathbb{N}$ に対して,
    \begin{itemize}
        \item $\Gamma \myalvdash{n} K_1 \myequiv{m} K_2$ かつ $\Gamma \myalvdash{n} K_2 \myequiv{m} K_3$ であるなら, $\Gamma \myalvdash{n} K_1 \myequiv{m} K_3$.
        \item $\Gamma \myalvdash{n} T_1 \myequiv{m} T_2$ かつ $\Gamma \myalvdash{n} T_2 \myequiv{m} T_3$ であるなら, $\Gamma \myalvdash{n} T_1 \myequiv{m} T_3$.
        \item $\Gamma \myalvdash{n} t_1 \myequiv{m} t_2$ かつ $\Gamma \myalvdash{n} t_2 \myequiv{m} t_3$ であるなら, $\Gamma \myalvdash{n} t_1 \myequiv{m} t_3$.
    \end{itemize}
    \begin{proof}
        \textbf{導出に関する構造帰納法による.}
    \end{proof}
\end{jlemma}

\begin{jlemma}[項の代入とアルゴリズム同値]\label{AlEqTermSubstitution}
    任意の $n \in \mathbb{N}$ に対して, 
    \begin{itemize}
        \item $\Gamma \myvdash{n} K_1 \myequiv{0} K_2$ であるなら, $\Gamma \myvdash{n} [x \mapsto t] K_1 \myequiv{0} [x \mapsto t]K_2$.
        \item $\Gamma \myvdash{n} T_1 \myequiv{0} T_2$ であるなら, $\Gamma \myvdash{n} [x \mapsto t] T_1 \myequiv{0} [x \mapsto t]T_2$.
        \item $\Gamma \myvdash{n} t_1 \myequiv{0} t_2$ であるなら, $\Gamma \myvdash{n} [x \mapsto t] t_1 \myequiv{0} [x \mapsto t]t_2$.
    \end{itemize}
    \begin{proof}
        \textbf{同値判断の導出に関する構造機能法による.}
    \end{proof}
\end{jlemma}
