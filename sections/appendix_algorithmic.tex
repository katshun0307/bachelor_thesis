
\vspace{10pt}

\begin{jdefinition}[項のレヴェル]\label{DefLevelTerms}
    自然数$n$に対して, 項のレヴェルを以下のように定める.
    \begin{itemize}
        \item レヴェル$0$の項は, 型の内部に存在しない項である.
        \item レヴェル$n$の項の中の型の中に含まれる項は, レヴェル$n + 1$ の項である.
    \end{itemize}
\end{jdefinition}

\begin{jdefinition}[アルゴリズム簡約]\label{DefAlgorithmicReduction}
    インデックス $m \in mathbb{N}$ を持つアルゴリズム簡約 $\alarrow{m}$ を,
    $m$ インデックスを持つアルゴリズム簡約規則の, 任意のレヴェルの項への適用とする.  
\end{jdefinition}

\begin{jdefinition}[アルゴリズム正規形]\label{DefANF}
    項 $t$ の インデックス$m$を持つ \emph{アルゴリズム的正規形} を,
    インデックス$m$ のアルゴリズム簡約による正規形とし, \anf{m}{t} と表記する.
\end{jdefinition}

\begin{jlemma}[レヴェル0のアルゴリズム簡約の強正規化性]\label{AlgorithmicSN0}
    任意の $m$に対して, インデックス $m$ のレヴェル$0$ の項の簡約は, 強正規化性を持つ.
    \begin{proof}
        \Cref{SN} より明らか.
    \end{proof}
\end{jlemma}

\begin{jtheorem}[アルゴリズム簡約の強正規化性]\label{AlgorithmicSN}
    任意の $m$に対して, インデックス$m$ でのアルゴリズム簡約は, 強正規化性を持つ.
    \begin{proof}
        \textbf{レヴェル$n$に関する帰納法による.}
        レヴェル $0$ の項に対しては, \Cref{AlgorithmicSN0} より成立. 
        レヴェル $i$ について示されれば, レヴェル$i+1$ についても示される.
        % We can prove
        % level $i+1$ subterm is strongly normalizing if we can prove for level $i$.
    \end{proof}
\end{jtheorem}

\begin{jtheorem}[アルゴリズム簡約の合流性]\label{AlgorithmicConfluence}
    $0$-インデックスのアルゴリズム簡約は, 全てのレヴェルの項について合流性を持つ.
    % The algorithmic reduction with index 0 is confluent for reduction on all levels.
    レヴェル $i$ に関する項の簡約で, $q \iarrow{0} r$ かつ, レヴェル$j$に関する項の簡約で $q \iarrow{0} s$ ならば,
    $r \iarrow{0} t$ and $s \iarrow{0} t$ であるような項 $t$ が存在する.
    \begin{proof}
        \begin{itemize}
            \item{$i = j$}\\
            \Cref{ChurchRosser} より, $t$ は存在する.
            \item{$i \neq j$}\\
            あるレヴェルの項での簡約は, 他のレヴェルの項について簡約基を生成しない.
            よって, 2つの簡約の順番の入れ替えによっても, 簡約結果は同じになる.
            % The reduction at a given level does not interfere with other terms of other levels; it does not create an another redex at a different level.
            % Thus, changing the order of the 2 reductions will yield the same result.
        \end{itemize}
    \end{proof}
\end{jtheorem}

\begin{jtheorem}[アルゴリズム簡約の型保存]\label{AlgorithmicTypePreservation}
    $\Gamma \myvdash{n} t \COL T$ かつ $t \alarrow{m} t'$ であるなら, $\Gamma \myvdash{n} t' \COL T$.
    \begin{proof}
        \textbf{レヴェル$n$に関する完全帰納法による.}.
        レヴェル$n-1$ までの項について示されているなら,
        \fauxsc{QT-App} と \fauxsc{T-Conv} を用いて,
        レヴェル$n$ の項についても示すことができる
    \end{proof}
\end{jtheorem}

\begin{jlemma}[アルゴリズム簡約と項の同値]\label{AlgorithmicTermEquivalence}
    $\Gamma \myvdash{n} M \COL T$ かつ $M \alarrow{m} M'$ であるなら, $\Gamma \myvdash{n} M \myequiv{m} M'$である.
    \begin{proof}
        \textbf{$M \alarrow{m}  M'$ の導出の関する帰納法による.}
    \end{proof}
\end{jlemma}

\begin{jtheorem}[アルゴリズム正規形と項の同値]\label{AlgorithmicANFEquivalence}
    $\Gamma \myvdash{n} t \COL T$ ならば, $\Gamma \myvdash{n} t \myequiv{m} \anf{m}{t}$ である.
    \begin{proof}
        \Cref{AlgorithmicTermEquivalence} と \Cref{AlgorithmicTypePreservation} から示される.
    \end{proof}
\end{jtheorem}

\begin{jlemma}[アルゴリズム同値の反射律]\label{AlRefl}
    任意の $n \in \mathbb{Z}, m \in \mathbb{N}$, に対して,
    \begin{itemize}
        \item $\Gamma \myalvdash{n} K \myequiv{m} K$.
        \item $\Gamma \myalvdash{n} T \myequiv{m} T$.
        \item $\Gamma \myalvdash{n} t \myequiv{m} t$.
    \end{itemize}
    \begin{proof}
        \textbf{導出に関する構造帰納法による.}
    \end{proof}
\end{jlemma}

\begin{jlemma}[アルゴリズム同値の対称律]\label{AlSym}
    任意の $n \in \mathbb{Z}, m \in \mathbb{N}$ に対して,
    \begin{itemize}
        \item $\Gamma \myalvdash{n} K_1 \myequiv{m} K_2$ であるなら, $\Gamma \myalvdash{n} K_2 \myequiv{m} K_1$.
        \item $\Gamma \myalvdash{n} T_1 \myequiv{m} T_2$ であるなら, $\Gamma \myalvdash{n} T_2 \myequiv{m} T_1$.
        \item $\Gamma \myalvdash{n} t_1 \myequiv{m} t_2$ であるなら, $\Gamma \myalvdash{n} t_2 \myequiv{m} t_1$.
    \end{itemize}
    \begin{proof}
        \textbf{導出に関する構造帰納法による.}
    \end{proof}
\end{jlemma}

\begin{jlemma}[アルゴリズム同値の推移律]\label{AlTrans}
    任意の $n \in \mathbb{Z}, m \in \mathbb{N}$ に対して,
    \begin{itemize}
        \item $\Gamma \myalvdash{n} K_1 \myequiv{m} K_2$ かつ $\Gamma \myalvdash{n} K_2 \myequiv{m} K_3$ であるなら, $\Gamma \myalvdash{n} K_1 \myequiv{m} K_3$.
        \item $\Gamma \myalvdash{n} T_1 \myequiv{m} T_2$ かつ $\Gamma \myalvdash{n} T_2 \myequiv{m} T_3$ であるなら, $\Gamma \myalvdash{n} T_1 \myequiv{m} T_3$.
        \item $\Gamma \myalvdash{n} t_1 \myequiv{m} t_2$ かつ $\Gamma \myalvdash{n} t_2 \myequiv{m} t_3$ であるなら, $\Gamma \myalvdash{n} t_1 \myequiv{m} t_3$.
    \end{itemize}
    \begin{proof}
        \textbf{導出に関する構造帰納法による.}
    \end{proof}
\end{jlemma}