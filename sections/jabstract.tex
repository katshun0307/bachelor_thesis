% japanese abstract (for 2 pages)

% == 背景 == %
% 多段階計算
\emph{多段階計算}は, 計算にステージという概念を導入し, プログラムコードを値として扱うことができる計算体系を指す.
Davies は, 与えられた項に対してその項のコードを表す演算子 \tnext と,
与えられたコードの中身を取り出して用いる演算子\tprev{} を備えた
多段階計算 \lamcirc を考案した.
この体系での特徴的な計算の例として, コードの埋め込みがある. 
コードの埋め込みとは, \tnext{} で表されたコードの中身に,
\tprev{} を適用した別のコード用いることで, 
コードの内部に別のコードを埋め込むことである.
これを利用し, コードの生成や評価を自在に行うことできるのがこの計算の強みである.

% 多段階計算の型システムが保証していること
Davies による \lamcirc には, 任意の型$T$に対して, ${\bigcirc}T$ という型が存在する.
これは, 型$T$ を持つ項のコードを表す型である.
すなわち, 型${\bigcirc}T$を持つコードは, \tprev によって
型$T$を持つ項となることを示しており,
このように型付けされた項は, プログラム中でコードの生成, 評価を行う際にも型エラーを発生しないことを保証している.

% 目的
本研究では, この型システムを拡張し, 保証できる性質をより強力にすることを目指す.
具体的には, Daviesによる \lamcirc に依存型と同値型を導入する.

% 依存型
\emph{依存型}は内部に項を持つ型のことであり, 
これを用いることで, 型の表現力が上がり, 一般的な型システムでは検出できないエラーを
型検査によって静的に検出できるようになる.
例として, 行列を表す型に行列の大きさに関する情報を持たせることによる,
次元の制約を満たさない行列積計算の検出が挙げられる.

% 同値型 
この依存型を利用した型の1つが Martin-Löf による\emph{同値型}である.
同値型は, 同じ型を持つ項を二つ内部に持ち,
この二つの項が等しいということを表現する型である.
同値型に関する規則は, 一般的な形をしており,
依存型を含んでいれば, およそどのような体系にも,
各々1つの導入規則と除去規則のみで導入可能である.
また, 体系で同値である項は全て同値型で表現できるので, 
体系の同値性と密接に関連している.

% == 方法(結果) == %
% 方法の概要
本研究では, まず, \lamcirc を, Harper らによる \lamlf と 同値型で拡張した体系を考える.
その後, 拡張した体系において同値性を定義する.
Kawata らは既に
\lamcirc を拡張した体系である \lamtriper を, \lamlf をもとに依存型で拡張した \lammd を考案しており,
本研究は, \lammd の体系の一部を同値型で拡張したものに相当すると考えられる.

% 簡約規則とステージ
体系における同値性を定義する際には, 簡約規則について考慮する必要がある.
簡約規則には, \mbeta-簡約, ステージに関する簡約, 
及び同値型に関する簡約がある.
しかし, プログラムコードの中身については, 
コードの埋め込みなどのステージに関する簡約
以外を行いたくない.
Yuse と Igarashi は, 簡約に, \tnext と \tprev の入り込み具合, 
すなわち評価対象の項がどのステージのものであるかに対応する
インデックスを導入し,
この値によって適用できる簡約規則を制限することで, 意図した場所にのみ簡約を
行えるようにした. 本研究でも, これと同様の方法で簡約規則を導入した.

% プログラムコード間の同値
以上のような簡約に基づいて, コード間の同値について考える.
コード内部の項には, 
適用できる簡約が制限されているため, それに応じて適用できる同値規則を制限したい.
% ステージに関する簡約以外は許されていないことから,
% 内部の項を\mbeta-同値などの通常の同値規則によって比較するべきではなく,
% 適用可能な簡約に対応する同値規則のみを用いたい.
このような定義を可能にするために, 簡約に関するインデックスと同様, 
同値の定義を整数インデックスについて拡張した.
このインデックスの値によって使用できる同値規則を制限することで, 
意図通りの同値性の定義を可能にする.

% 証明したもの
加えて, 提案した体系が安全であることを示すために, 1ステップの簡約によって項の持つ型が変わらないことを示す型の保存,
型付けされた項は値であるか, それに対して1ステップの簡約が存在することを示す進行を示した.
さらに, 型付けされた項には無限長の簡約列が存在しないことを示す強正規化性,
項はどのような簡約経路を辿っても再び同じ項に合流する性質である合流性を示す.
最後に, 決定的に型を与えることができるアルゴリズム的型付け規則を導入し,
これが通常の型付け規則に対して健全かつ完全であることを証明する.

% 今後の課題 (証明支援系の話)
% 対応する論理体系(自然演繹)についての今後の課題
本研究では, 提案した型システムが,
どのような論理体系に対応しているかについては述べていない.
依存型を備えたラムダ計算は, カリーハワード同型対応を通じて,
直観主義述語論理に対応することが知られているが,
多段階計算と同値型によって, 対応する論理体系は述語論理からどのように拡張され,
どのような公理系を用いて表されるのかについては, 今後の課題としたい.
