% japanese abstract (for 2 pages)

% == 背景 == %
% 多段階計算
\emph{多段階計算}は, 計算にステージという概念を導入し, プログラムコードを値として扱うことができる計算を指す.
Davies は, 与えられた項に対してその項のプログラムコードを表す\textrm{next}と,
与えられたプログラムコードの項を計算する\textrm{prev}を持つような多段階計算の体系$\lambda{\bigcirc}$を考案した.
これらを利用し, コードの生成, 埋め込みや評価を自在に行うことできるのがこの計算の強みである.

% 多段階計算の型システムが保証していること
Davies による多段階計算体系には, 任意の型$T$に対して, ${\bigcirc}T$ という型が存在する.
型${\bigcirc}T$を持つ項は, \textrm{prev} によって評価されると, 型$T$の項となることを示しており, 
型付けされた項は, プログラム中でコードの生成, 評価を行う際にも型エラーを発生しないことを保証している.

% カリーハワード対応
% \emph{カリーハワード同型対応}は, 型付きラムダ計算と, 直観主義論理体系間の対応関係であり, 
% 計算における項は論理体系における証明に, 型は命題に対応していることが知られている.
% この性質からラムダ計算は, 論理を扱う証明支援系に応用できる.

% 目的
本研究では, この型システムを拡張し, 保証できる性質をより強力にすることを目指す.
具体的には, Daviesによる$\lambda{\bigcirc}$に依存型と同値型を導入する.

% 依存型 (説明っぽさをなくす)
\emph{依存型}は内部に項を持つ型のことであり, 
これを用いることで, 型の表現力が上がり, 一般的な型システムでは検出できないエラーを
型検査のよって静的に検出できるようになる.
例として, 行列を表す型に行列の大きさに関する情報を持たせることによる, 
大きさの一致しない行列積計算の検出が挙げられる.
% さらには, 依存型はカリーハワード同型対応を通じて
% 述語とみなすことができ, Coq などの証明支援系に利用されている.

% 同値型 
この依存型を利用した型の1つが Löf による\emph{同値型}である.
同値型は, 同じ型を持つ項を二つ受け取り, この二つの項が等しいということを表現する型である.
同値型は, 各々1つの導入規則と除去規則のみであらゆる型に関する同値を表現できる.
また, 2つの項が同値型を持つためには, 2つの項が同値である必要があるため, 
同値型は体系の同値の定義と密接に関連する.
% また, 同値型をもつ項は, カリーハワード同型対応を通じて, 同値性の証明とも見ることができる.
% このような性質から, 証明支援系への応用が考えられる.

% == 目的 == %
% 本研究の目的は, 多段階ラムダ計算に
% 依存型と同値型という証明支援系において利用できる言語機能を付け加えることで, 
% 多段階計算の強みであるプログラムコードに関しても証明を行うことができる体系を目指す.

% == 方法(結果) == %
% 方法の概要
本研究では, 
Davies による多段階ラムダ計算の体系を
Harper らによる依存型を持つ体系である Edinburgh Logical Framework について拡張し,
同値型の規則に関する規則を加え. 
% Kawata による$\lambda^{\textrm{MD}}$ を参考にし, 
この体系の項に対して同値の定義を試みた.

% プログラムコード間の同値
項に関する同値のうち, 多段階計算によってもたらされるプログラムコードについて同値を比較する際には, 
中身の項を通常の同値で定義するのではなく, 中身の項の構文木を用いて同値を定義したい. 
しかしながら, プログラムコードを表す項の中の値に値の埋め込みがある場合は,
埋め込みを展開した状態で同値を評価したい.
以上のことを考慮した同値の評価を可能にするために, 
同値記号に\textrm{next}, \textrm{prev}の入り組み具合を表す整数インデックスを付加した.
このインデックスによって, 使用できる同値の評価規則が制限され, 意図通りの同値の定義を
可能にする.

% 簡約規則とステージ
さらには, この体型に1ステップ簡約の形で簡約規則を導入する.
簡約規則は\mbeta-簡約, ステージに関する簡約, 及び同値型に関する簡約がある.
しかし, プログラムコードの中身については, プログラムコードへの埋め込みの内部を除き, 
\mbeta-簡約など, 行いたくない簡約が存在する.
このため, 簡約の矢印についても項の同値と同様,
\textrm{next}, \textrm{prev}の入り組み具合を表す自然数インデックスを導入した.
このインデックスの値によって適用できる簡約規則を制限し, 意図した場所にのみ簡約を
行えるようにした.

% 証明したもの
加えて, 1ステップの評価によって項の持つ型が変わらないことを示す型の保存,
型付けされた項は値であるか, それに対して1ステップの簡約が存在することを示す進行を示し, 
提案した体系が安全であることを示した.
さらに自由変数に対する同じ型を持つ項の代入は型は保存することを示す代入補題,
型付けされた項には無限長の簡約列が存在しないことを示す強正規化性,
項はどのような簡約経路を辿っても再び同じ項に合流する性質である合流性を示す.
最後に, 決定的に型を与えることができるアルゴリズム的型付け規則を導入し, 
これが通常の型付け規則に対して健全かつ完全であることを証明する.

% 今後の課題 (証明支援系の話)
依存型を備えたラムダ計算は, カリーハワード同型対応を通じて, 
直観主義述語論理に対応することが知られており, 証明支援系において用いられている.
また, 本研究で新たに導入した同値型も, 2つの項の同値という命題に対応しているとみなせ, 
証明支援系における応用が考えられる. 本研究の要素である多段階計算と同値型によって, 
対応する論理体系は述語論理からどのように拡張されるのかについては,
今後の課題としたい.
