% japanese abstract (for 2 pages)

% == 背景 == %
% 多段階計算
\emph{多段階計算}は, 計算にステージという概念を導入し, プログラムコードを値として扱うことができる計算を指す.
Davies は, 与えられた項に対してその項のプログラムコードを表す\textrm{next}と,
与えられたプログラムコードの項を計算する\textrm{prev}を持つような多段階計算の体系を考案した.
これらを利用し, コードの生成, 埋め込みや評価を自在に行うことできるのがこの計算の強みである.

% カリーハワード対応
\emph{カリーハワード同型対応}は, 型付きラムダ計算と, 直観主義論理体系間の対応関係であり, 
計算における項は論理体系における証明に, 型は命題に対応していることが知られている.
この性質からラムダ計算は, 論理を扱う証明補助システムに応用できる.

% 依存型
また, \emph{依存型}は内部に項を持つ型のことを言う. 
これを用いることで, 型の表現力が上がり, 一般的な型システムでは検出できないエラーを
型検査のよって静的に検出できるようになる.
例として, 行列を表す型に行列の大きさに関する情報を持たせることによる, 
大きさの一致しない行列積計算の検出が挙げられる.
さらには, 依存型はカリーハワード同型対応を通じて命題の引数として
証明をとると考えることができ, Coq などの証明補助システムに利用されている.
 
% 同値型
この依存型を利用した型の1つに\emph{同値型}\cite{Lof1984}が存在する.
同値型は, 同じ型を持つ項を二つ受け取り, この二つの項が等しいということを表現する.
また, 同値型をもつ項は, カリーハワード同型対応を通じて, 同値性の証明とも見ることができる.
このような性質から, 証明補助システムへの応用が考えられる.

% == 目的 == %
本研究の目的は, 多段階ラムダ計算に
依存型と同値型という証明補助システムにおいて利用できる言語機能を付け加えることで, 
多段階計算の強みであるプログラムコードに関しても証明を行うことができる体系を目指す.

% == 方法(結果) == %
% 方法の概要
本研究では, (可能世界が線形に連なっているモデルに対応する様相論理に対応する)
多段階ラムダ計算の体系\cite{Davies1996a}に
依存型を持つ体系である Edinburgh Logical Framework\cite{Harper1993}を導入し, 
この体系の項に対して同値を試みる.

% プログラムコード間の同値
項に関する同値のうち, 多段階計算によってもたらされるプログラムコードについて同値を比較する際には, 
中身の項を通常の同値で定義するのではなく, 中身の項の構文木を用いて同値を定義したい. 
しかしながら, プログラムコードを表す項の中の値に値の埋め込みがある場合は,
埋め込みを展開した状態で同値を評価したい.
以上のことを考慮した同値の評価を可能にするために, 
同値記号に\textrm{next}, \textrm{prev}の入り組み具合を表す整数インデックスを付加した.
このインデックスによって, 使用できる同値の評価規則が制限され, 意図通りの同値の定義を
可能にする.

% 型の間の同値
% さらに型の同値についても, 依存型を含む体系であるので,型が内部に持ち得る項の同値についても考える必要がある.
% しかし, このような場合の項の同

% 簡約規則とステージ
さらには, この体型に1ステップ簡約の形で簡約規則を導入する.
簡約規則は\mbeta-簡約, ステージに関する簡約, 及び同値型に関する簡約がある.
しかし, プログラムコードの中身について, \mbeta-簡約などの簡約を行いたくない.
このため, 簡約の矢印についても項の同値と同様, 自然数インデックスを導入した.
このインデックスの値によって適用できる簡約規則を制限し, 意図した場所にのみ簡約を
行えるようにした.

最後に, 決定的に型を与えることができるアルゴリズム的型付け規則を導入し, 
これが通常の型付け規則に対して健全かつ完全であることを証明する.
1ステップの評価によって項の持つ型が変わらないことを型の保存,
型付けされた項は値であるか1ステップの簡約が存在することを表す進行,
さらに同じ型を持つ別の項を部分項に対して代入しても型は保存することを示す代入補題,
無限長の簡約列が存在しないことを示す強正規化性,
同じ項はどのような簡約経路を辿っても再び同じ項に合流する性質である合流性を示す.
