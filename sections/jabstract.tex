% japanese abstract (for 2 pages)

% == 背景 == %
% 多段階計算
\emph{多段階計算}は, プログラムコードを値として扱うことができる計算を指す.
与えられた項に対してその項のプログラムコードを表す\textrm{next}と,
与えられたプログラムコードの項を計算する\textrm{prev}を持ち,
コードの生成, 埋め込みや評価を自在に行うことできるのがこのような計算の強みである.

% カリーハワード対応
\emph{カリーハワード同型対応}は, 型付きラムダ計算と, 直観主義論理体系間の対応関係であり, 
計算における項は論理体系における証明に, 型は命題に対応していることが知られている.
この性質からラムダ計算は, 論理を扱う証明補助システムにおいて用いられている.

% 依存型
また, \emph{依存型}は内部に項を持つ型のことを言う. 
これを用いることで, 型の表現力が上がり, 一般的な型システムでは検出できないエラーを
型検査のよって静的に検出できるようになる.
例として, 行列を表す型に行列の大きさに関する情報を持たせることで, 
大きさの一致しない行列積計算の検出が挙げられる.
さらには, 依存型はカリーハワード同型対応を通じて命題の引数として
証明をとると考えることができ, Coq を始めとする証明補助システムに利用されている.
 
% 同値型
この依存型を利用した型の1つに\emph{同値型}\cite{Lof1984}が存在する.
同値型は, 同じ型を持つ項を二つ受け取り, この二つの項が等しいということを表現する.
同値型をもつ項は, カリーハワード同型対応を通じて, 同値性の証明とも見ることがでる.
このような性質から,同値型は証明補助システムなどへ応用されてきた.
また, 同値型によって表現できるのは, 項の同値だけであるので, 体系での同値の定義に密接に関連する.

% == 目的 == %
本研究の目的は, 多段階ラムダ計算に依存型と合わせて同値型を付け加えることで, 
多段階計算の表現を,
依存型や同値型などの証明補助システムにおいて利用できる言語機能と組み合わせることで,
多段階計算の強みであるプログラムコードに関して証明を行うことができる体系を目指す.

% == 方法(結果) == %
本研究では, (可能世界が線形に連なっているモデルに対応する様相論理に対応する)
多段階ラムダ計算の体系\cite{Davies1996a}に
依存型を持つ体系である Edinburgh Logical Framework\cite{Harper1993}を導入し, 
この体系の項に対して同値を形式的に定義する. さらに同値型に関する規則を追加する.
また, 決定的に型を与えることができるアルゴリズム的型付け規則を導入し, 
通常の型付け規則に対して健全かつ完全であることを証明する.
続いて, 体系の簡約規則を与え, 
1ステップの評価によって項の持つ型が変わらないことを型の保存,
型付けされた項は値であるか1ステップの簡約が存在することを表す進行,
さらに同じ型を持つ別の項を部分項に対して代入しても型は保存することを示す代入補題,
無限長の簡約列が存在しないことを示す強正規化性,
同じ項はどのような簡約経路を辿っても再び同じ項に合流する性質である
合流性を示す.
