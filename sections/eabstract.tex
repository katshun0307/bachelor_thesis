
% == background == %
% multi-stage calculus
\emph{Multi-stage Computation} introduces an idea of stages to computation, which enables 
manipulation of program code as values.
Davies introduced \lamcirc, which included terms of \textrm{next}, which enables handling a term 
as a code, and \textrm{prev}, which reduces terms inside code to normal terms.
One of the key features of this system is the embedding of terms, which embeds
code value inside other code values by applying \textrm{prev} to the code values to embed.
It is the strength of this system that enables generation and evaluation of code at will. 

% what the type system guarantees
In Davies's \lamcirc, there exists type $\bigcirc T$ for an arbitrary type $T$.
Type $\bigcirc T$ shows that a term with this type will have type $T$ when it is applied to \textrm{prev},
thus guaranteeing that no error will occur during generation and evaluation of code at run-time.

% == target == %
This research aims at extending the type system of \lamcirc to be as strong as possible,
namely introducing dependent types and equality types into the system.

% dependent types
\emph{Dependent types} refer to a type that contains program terms. 
By using dependent types, one can detect potential runtime errors that cannot be detected with usual type systems,
for example detecting matrix multiplication with incompatible dimensions 
by using dependent types with dimension information for matrices.

% equality types
One example of the application of dependent types are \emph{equality types}, introduced by Martin-Löf.
An equality type contains 2 terms of the same type, representing that the 2 terms are equivalent.
Typing rules for equality types have a general form, enabling us to easily extend the system
for almost any system which includes dependent types.

% research method
In this research, we extend \lamcirc by Davies by adding dependent types and equality types,
referring to \lamlf by Harper et al. for dependent types. 
Kawata has already proposed \lammd, which extends \lamcirc with CSP and \lamlf.
Therefore, the system proposed in research can be understood as the extension of \lamcirc without CSP
with equality types.

% reduction rules and stages
The reductions in this system include \mbeta-reduction, reduction on stages, and reduction on equality types.
However, we do not want to perform reductions inside code, except for reductions on stages.
Yuse and Igarashi have enabled performing appropriate reductions on terms by extending reductions with
index numbers, representing how deep the current term is nested under \textrm{next} and \textrm{prev}.
We have taken a similar approach in this research.

% equality between codes
Also, when comparing equivalence between code, 
we do not want to judge equivalence based on normal equivalence rules such as \mbeta-reduction,
because only stage reduction is allowed inside code. Instead, we want to judge equivalence based on
the corresponding equivalence rules allowed.
For enabling this kind of definition, we expanded the definition of equivalence by adding an index number, 
just as we did in defining reduction.
By limiting the available equivalence rules with this index, 
we make it possible to perform reduction as intended.

% what we have proved
Furthermore, to prove the safety of our proposed system, 
we prove preservation, stating that a type of a term will not change on 1-step reduction,
and also progress, stating that either a well-typed term is a value, or there exists a step of reduction.   
We also prove strong normalization, stating that a well-typed term will not have an infinite sequence of reduction steps,
and confluence of terms under reduction, which shows that there can be multiple reduction paths that yield the same result.
Finally, we introduce algorithmic typing to type terms decidably,
and prove that algorithmic typing is sound and complete with regards to the normal typing rules.

% Future Work
In this research, we have not been able to state what kind of logical system corresponds to 
the type system we have introduced. We know that lambda calculus with dependent types corresponds to
the intuitionistic propositional logic, but it is left for future work of how the logic will extend with
multi-stage computation and equality types.
