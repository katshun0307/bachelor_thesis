
% previous section: ./property.tex

% 章の構成
この章では, 
アルゴリズム的型付けを導入した後, この型付け規則が通常の型付け規則に対して健全かつ完全であること, 
即ち, アルゴリズム的型付けで導出できる判断は通常の型付け規則でも導出可能であることと, 
通常の型付け規則で導出可能である判断は, アルゴリズム的型付けで導出可能であることを示す.

\subsection{アルゴリズム的型付け規則}

アルゴリズム的型付け規則とは, 環境と項/型が与えられたときに,
適用できる型付け/カインド付け規則が一意, あるいは有限個に定まり,
有限時間で, ふさわしい型/カインド(のうち一つ) を導出できるような体系のことを言う.
\Cref{fig_al_q} の \anf{m}{t} は, 後に説明される. また, $\equiv_{\alpha}$ は, \malpha -同値を示す.

\Cref{fig_al_f,fig_al_k,fig_al_t,fig_al_qk,fig_al_qt,fig_al_q} は, 提案するアルゴリズム的型付け規則である.
% \Cref{fig_al_q} における \textrm{ANF} は, \Cref{subsection_al_proof_before} で定義する.

\FB
\begin{fig}{アルゴリズム的適格性規則}
    \centering
    \footnotesize
    % \WFAEmpty \bcpnl
    % \WFATm \quad
    % \WFATy \bcpnl
    \WFAStar \quad
    \WFAPi
    \label{fig_al_f}
\end{fig}

\begin{fig}{アルゴリズム的カインド付け規則}
    \centering
    \footnotesize
    \KAVar \quad
    \KAPi \bcpnl
    \KAApp \bcpnl
    \KAEquiv
    \label{fig_al_k}
\end{fig}
    
\begin{fig}{アルゴリズム的型付け規則}
    \centering
    \footnotesize
    \TAVar\quad
    \TAAbs
    \bcpvspace\\
    \TAApp \bcpvspace\\
    \TAPrevIntro \quad
    \TANextIntro
    \bcpvspace \\
    \TAEqIntro \bcpvspace\\
    \TAEqElim
    \label{fig_al_t}
\end{fig}

\begin{fig}{アルゴリズム的カインド同値規則}
    \centering
    \footnotesize
    \QKAStar \bcpnl
    \QKAPi
    \label{fig_al_qk}
\end{fig}

\begin{fig}{アルゴリズム的型同値規則}
    \centering
    \footnotesize
    \QTAVar \quad
    \QTAPi \bcpnl
    \QTAApp \bcpnl
    \QTACircIntro
    \label{fig_al_qt}
\end{fig}

\begin{fig}{アルゴリズム的項同値規則}
    \centering
    \footnotesize
    \QAAnf
    \label{fig_al_q}
\end{fig}
\FB

\subsection{アルゴリズム的型付け規則の証明の準備}\label{subsection_al_proof_before}

アルゴリズム的型付けの正しさを示すために, 
上で定義したアルゴリズム的型付けが, 通常の型付け規則に関して健全かつ完全であることを証明する.

はじめに, アルゴリズム的簡約を定義する. アルゴリズム的簡約とは,
\Cref{FigAlgorithmicReduction} で示される簡約を, 全てのレヴェルの項(\Cref{DefLevelTerms}参照)に対して行い, 
take the congruence on terms (訳???)したものである.

\begin{fig}{アルゴリズム的簡約規則}
    \centering
    \infax[\fauxsc{EA-Beta}]{
        (\mylam{x}{T}{s})t \iarrow{m} [x \mapsto t]s 
        \andalso m \le 0
    } \bcpnl
    \infax[\fauxsc{EA-StagedBeta}]{
        \myprev{\mynext{t}} \iarrow{m} t
        \andalso m \le 0
    } \bcpnl
    \infax[\fauxsc{EA-StagedEta}]{
        \mynext{\myprev{t}} \iarrow{m} t
        \andalso m \le 1
    } \bcpnl
    \infax[\fauxsc{EA-Equal}]{
        \myidpeel{\myid{a}}{x}{t} \iarrow{m} [x \mapsto a] t
        \andalso m = 0
    }
    \label{FigAlgorithmicReduction}
\end{fig}
\FB

このアルゴリズム的簡約に対して, アルゴリズム的正規形を定義する.

\FB
\begin{jdefinition}[アルゴリズム正規形]\label{DefANF}
    項 $t$ の インデックス$m$を持つ \emph{アルゴリズム的正規形} を,
    インデックス$m$ のアルゴリズム簡約による正規形とし, \anf{m}{t} と表記する.
\end{jdefinition}
\FB

このアルゴリズム的正規形の一意性は, \Cref{AlgorithmicANFUniqueness} で示される.

\begin{jtheorem}[アルゴリズム正規形の一意性]\label{AlgorithmicANFUniqueness}
    任意の型付けされた項に対して, インデックス $m$ をもつアルゴリズム正規形は一意である.
    \begin{proof}
        \Cref{AlgorithmicSN} と \Cref{AlgorithmicConfluence} から従う.
    \end{proof}
\end{jtheorem}

\FB

\subsection{アルゴリズム的型付け規則の正しさ}\label{subsection_al_proof}

% 健全性
はじめに, アルゴリズム的型付け規則の健全性を示していく.

\begin{jtheorem}[アルゴリズム的同値の宣言的同値に対する健全性]\label{SoundAQ}
    任意の $n \in \mathbb{N}, m \in \mathbb{Z}$ に対して,
    \begin{itemize}
        \item   ${\Gamma \myalvdash{n} T_1 \myequiv{m} T_2}$, 
                ならば ${\Gamma \myvdash{n} T_1 \myequiv{m} T_2}$.
        \item   ${\Gamma \myalvdash{n} K_1 \myequiv{m} K_2}$, 
                ならば ${\Gamma \myvdash{n} K_1 \myequiv{m} K_2}$.
        \item   ${\Gamma \myalvdash{n} s \myequiv{m} t \COL T}$ 
                ならば ${\Gamma \myvdash{n} s \myequiv{m} t \COL T}$
    \end{itemize}
    \begin{proof}
        \textbf{導出に関する構造帰納法による.}
        \begin{itemize}
            \item{\fauxsc{QA-Anf}}\\
            \fauxsc{QA-Anf} より, ${\anf{m}{s} \equiv_{\alpha} \anf{m}{t}}$ を得る.
            よって, ${\Gamma \myvdash{n} \anf{m}{s} \myequiv{m} \anf{m}{t}}$.
            また, \Cref{AlgorithmicANFEquivalence} より,
            ${\Gamma \myvdash{n} s \myequiv{m} \anf{m}{s}}$, ${\Gamma \myvdash{n} t \myequiv{m} \anf{m}{t}}$ が得られる.
            以上より, ${\Gamma \myvdash{n} s \myequiv{m} t}$ を得る.
        \end{itemize}
    \end{proof}
\end{jtheorem}

\begin{jtheorem}[アルゴリズム的型付け及びカインド付けの健全性]\label{SoundAKT}
    任意の ${n \in \mathbb{N}}$,
    \begin{itemize}
        \item ${\Gamma \myalvdash{n} K}$, ならば, ${\Gamma \myvdash{n} K}$. 
        \item ${\Gamma \myalvdash{n} T \DCOL K}$, ならば, ${\Gamma \myvdash{n} T \COL K}$. 
        \item ${\Gamma \myalvdash{n} t \COL T}$ ならば, ${\Gamma \myvdash{n} t : T}$.
    \end{itemize}
    \begin{proof}
    \textbf{導出に関する構造帰納法による.}
        \begin{itemize}
            \item{\fauxsc{KA-App}}\\
            帰納法の仮定と \Cref{SoundAQ} より, 以下の導出を得る.
            \begin{prooftree}
                \AxiomC{$\Gamma \myvdash{n} t \COL T_2$}
                \AxiomC{$\Gamma \myvdash{n} T_1 \myequiv{0} T_2$}
                \myplabel{T-Conv}
                \BinaryInfC{$\Gamma \myvdash{n} t \COL T_1$}
                \AxiomC{$\Gamma \myvdash{n} S \DCOL \Pi x \COL T_1. K$}
                \myplabel{K-App}
                \BinaryInfC{$\Gamma \myvdash{n} S \mysp t \DCOL [x \mapsto t] K$}
            \end{prooftree}
            \item{\fauxsc{TA-App}}\\
            以下の導出のより示される.
            \begin{prooftree}
                \AxiomC{$\Gamma \myvdash{n} S_1 \myequiv{n} S_2$}
                \AxiomC{$\Gamma \myvdash{n} t_2 \COL S_2$}
                \myplabel{T-Conv}
                \BinaryInfC{$\Gamma \myvdash{n} t_2 \COL S_1$}
                \AxiomC{$\Gamma \myvdash{n} t_1 \COL \Pi x \COL S_1. T$}
                \myplabel{T-App}
                \BinaryInfC{$\Gamma \myvdash{n} t_1 \mysp t_2 \COL [x \mapsto t_2] T$}
            \end{prooftree}
        \end{itemize}
        % KA-Pi, KA-Var, KA-Equiv
    \end{proof}
\end{jtheorem}

\begin{jcorollary}[アルゴリズム的型付けの健全性]
    アルゴリズム的型付けは, 宣言的な型付け規則に対して, 健全である.
    即ち, アルゴリズム的型付けによて得られた判断は, 通常の型付け規則によっても得られる.
    \begin{proof}
        \Cref{SoundAQ,SoundAKT}による.
    \end{proof}
\end{jcorollary}
\vspace{10pt}

% 完全性
次に, 完全性を示す.

\begin{jtheorem}[アルゴリズム的型付けの完全性]\label{AlgorithmicComp}
    任意の $n \in \mathbb{N}, m \in \mathbb{Z}$ に対して, $K, K'$ をカインド, $S, T$ を型, $s, t$ を項とするとき, 以下が成立する.
    \begin{itemize}
        \item $\Gamma \myvdash{n} K$ であるなら, $\Gamma \myalvdash{n} K$.
        \item $\Gamma \myvdash{n} T \DCOL K$ であるなら,
        $\Gamma \myalvdash{n} K$, $\Gamma \myalvdash{n} K \myequiv{0} K'$ かつ $\Gamma \myalvdash{n} T \DCOL K$
        が成立するような $K'$ が存在する.
        \item $\Gamma \myvdash{n} t \COL T$ であるなら,
        $\Gamma \myalvdash{n} T'$, $\Gamma \myalvdash{n} T \myequiv{0} T'$ かつ $\Gamma \myalvdash{n} t \COL T'$
        が成立するような $T'$ が存在する.
        \item $\Gamma \myvdash{n} K \myequiv{m} K'$ であるなら, $\Gamma \myalvdash{n} K \myequiv{m} K'$.
        \item $\Gamma \myvdash{n} S \myequiv{m} T$ であるなら, $\Gamma \myalvdash{n} S \myequiv{m} T$.
        \item $\Gamma \myvdash{n} s \myequiv{m} t$ であるなら, $\Gamma \myalvdash{n} s \myequiv{m} t$.
    \end{itemize}
    \begin{proof}
        \textbf{導出に関する構造帰納法による.}
        \begin{itemize}
            % kinding
            \item{\fauxsc{K-App}} ($\Gamma \myvdash{n} S t \DCOL [x \mapsto t] K_1$)\\
            ${\Gamma \myvdash{n} S \DCOL \Pi x \COL T_1. K_1}$ に帰納法の仮定を適用することにより,
            ${\Gamma \myalvdash{n} J \myequiv{m} \Pi x \COL T_1 . K} \cdots (1)$ かつ
            ${\Gamma \myalvdash{n} S \DCOL J \cdots (2)}$ であるような $J$ が存在することがわかる.
            $J$ を, $\Pi x \COL T_2. K_2$ とすると,
            (1) の導出は \fauxsc{QKA-Pi} によるので, $\Pi x \COL T_2. K_2 \equiv_{0} \Pi x \COL T_1. K'$,
            ${\Gamma \myalvdash{n} T_1 \myequiv{m} T_2 \cdots (3)}$ かつ ${\Gamma x \mysncln{n} T_1 \myalvdash{n} K_1 \myequiv{0} K_2 \cdots (4)}$ であるとわかる.
            また, ${\Gamma \myvdash{n} t_1 \DCOL T_1}$ に帰納法の仮定を適用することにより,
            $\Gamma \myvdash{n} T' \myequiv{0} T_1$ であるような ${\Gamma \myalvdash{n} t_1 \DCOL T'} \cdots (5)$ を得る.
            最後に, \fauxsc{K-App} と (3), (4), (5) を用いることで,
            ${\Gamma \myalvdash{n} S t \DCOL [x \mapsto t] K_2}$ を得る.
            \item{\fauxsc{K-Conv}}($\Gamma \myvdash{n} T \DCOL K$ から $\Gamma \myvdash{n} T \DCOL K'$,
            $\Gamma \myvdash{n} K \myequiv{0} K'$)\\
            帰納法の仮定より,
            ${\Gamma \myalvdash{n} T \DCOL J}$ かつ ${\Gamma \myalvdash{n} K \myequiv{m} J}$ であるような $J$ が存在する.
            また, ${\Gamma \myalvdash{n} K \myequiv{m} K'}$ である. アルゴリズム同値の推移律(\Cref{AlTrans}) より,
            $\Gamma \myalvdash{n} K' \myequiv{n} J$ を得る.
            % typing
            \item{\fauxsc{T-App}}($\Gamma \myvdash{n} t_1 t_2 \COL [x \mapsto t_2] T_1$)\\
            帰納法の仮定より, ${U \equiv \Pi x \COL S_2. T_2}$ とすると,
            ${\Gamma \myalvdash{n} \Pi x \COL S_1. T_1 \myequiv{m} U \cdots (1)}$ かつ ${\Gamma \myalvdash{n} U}$ であるような $U$ が存在する.
            (1) の導出は \fauxsc{QTA-Pi} によるので,
            ${\Gamma \myalvdash{n} S_1 \myequiv{m} S_2} \cdots (2)$ かつ ${\Gamma, x \mysncln{n} S_1 \myalvdash{n} T_1 \myequiv{m} T_2 \cdots (3)}$ である.
            \\また, $\Gamma \myvdash{n} t_2 \COL S_1$ に帰納法の仮定を適用することにより,
            ${\Gamma \myalvdash{n} S_1 \myequiv{m} S_3}$ かつ ${\Gamma \myalvdash{n} t_2 \COL S_3}$ であるような $S_3$ が存在する.
            最後に, \fauxsc{QTA-App} から $\Gamma \myalvdash{n} S_2 \myequiv{m} S_3$ を用いて示される.
            \item{\fauxsc{T-Conv}}($\Gamma \myvdash{n} t \COL T'$)\\
            帰納法の仮定より, $\Gamma \myalvdash{n} t \COL U$ であるような $U$ が 存在する.
            また, $\Gamma \myalvdash{n} T \myequiv{m} T'$ も成立する.
            \Cref{AlTrans} より, $\Gamma \myalvdash{n} U \myequiv{m} T'$ を得る.
            % kind equivalence
            % \item{\fauxsc{QK-Pi}}\\
            % By straightforward induction.
            \item{\fauxsc{QK-Refl}}\\
            \Cref{AlRefl} に従う.
            \item{\fauxsc{QK-Sym}}\\
            \Cref{AlSym} に従う.
            \item{\fauxsc{QK-Trans}}\\
            \Cref{AlTrans} に従う.
            % term equivalence
            \item{\fauxsc{Q-Abs}} ($\Gamma \myvdash{n} \mylam{x}{S_1}{t_1} \myequiv{m} \mylam{x}{S_2}{t_2}$)\\
            帰納法の仮定より, $\Gamma \myvdash{n} S_1 \myequiv{m} S_2, \mysp \Gamma, x \mysncln{n} S_1 \myvdash{n} t_1 \myequiv{m} t_2$ であり,
            さらに, \Cref{AlgorithmicANFEquivalence} より $\anf{m}{t_1}\myequiv{m} \anf{m}{t_2}$ である.
            よって, $\anf{m}{\mylam{x}{S_1}{t_1}} \alphaequiv \anf{m}{\mylam{x}{S_2}{t_2}}$
            \item{\fauxsc{Q-App}} ($\Gamma \myvdash{n} s_1 s_2 \myequiv{m} t_1 t_2$)\\
            \Cref{AlgorithmicANFEquivalence} より, 容易に示される.
            \item{\fauxsc{Q-Beta}} ($\Gamma \myvdash{n} (\mylam{x}{S}{t}) s \myequiv{m} [x \mapsto s] t \COL [x \mapsto s] T$)\\
            $(\mylam{x}{S}{t}) s \rarrow [x \mapsto s] t$ に \Cref{AlgorithmicTermEquivalence} を適用することで示される.
            \item{\fauxsc{Q-Refl}}\\
            \Cref{AlRefl} に従う.
            \item{\fauxsc{Q-Sym}}\\
            \Cref{AlSym} に従う.
            \item{\fauxsc{Q-Trans}}\\
            \Cref{AlTrans} に従う.
        \end{itemize}
    \end{proof}
\end{jtheorem}