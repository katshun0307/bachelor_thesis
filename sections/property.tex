
この章では, 提案した体系の安全性を示すため,  はじめに代入補題を示したあと, 
それを用いて保存と進行を示す. その後, 簡約規則の強正規化性と合流性を示す.

\subsection{体系の安全性}

はじめに, \Cref{Substitution} で, 代入補題を示す.

\begin{jlemma}[代入補題]\label{Substitution}
    $\Gamma, x\mysncln{n_1} S \myvdash{n_2} t \COL T$ かつ $\Gamma \myvdash{n_1} s \COL S$ であるとき, 
    $\Gamma \myvdash{n_2} [x \mapsto s] t \COL [x \mapsto s] T$ が成立する.
    \begin{proof}
        $\Gamma, x \mysncln{n_1} S \myvdash{n_2} t \COL T$ の導出に関する構造帰納法による.
        \begin{inditem}
            \item{\fauxsc{T-Var}}(t = y)
            \begin{enumerate}
                \item{y = x}\\
                $\Gamma, x \mysncln{n_1} \myvdash{n_2} x \COL T$ かつ $\Gamma \myvdash{n_1} s \COL S$ とする.
                環境内に同じ束縛を持つ変数は存在し得ないことから, 
                $\Gamma$ は, $x$ への束縛を有せず, $n_1 = n_2$ とわかる.
                また, $S$ は自由変数 $x$ を含まないため, $[x \mapsto s] S = S$ である.
                よって, $\Gamma \myvdash{n_1} s \COL S$ を示せば十分であり, 示された.
                \item{$y \neq x$}\\
                $\Gamma, x \mysncln{n_1} S \myvdash{n_2} y \COL T$ かつ $\Gamma \myvdash{n_1} s \COL S$ とする.
                \fauxsc{T-Var} より, $(y \mysncln{n_2} T) \in \Gamma$ である. 
                よって, $\Gamma \myvdash{n_2} y \COL T$. $T$ は自由変数 $x$ を含まないため, $[x \mapsto s] T = T$ である.
                よって, 示された.
            \end{enumerate}
            \item{\fauxsc{T-Abs}} ($t = \mylam{y}{S'}{t'}, \quad T = \Pi y \COL S'. T'$)\\ 
            \begin{prooftree}
                \AxiomC{$\Gamma, x \mysncln{n_1} S, y \mysncln{n_2} S' \myvdash{n_2} t' \COL T'$}
                \myplabel{T-Abs}
                \UnaryInfC{$\Gamma, x \mysncln{n_1} S \myvdash{n_2} \mylam{y}{S'}{t'} \COL \Pi y \COL S'. T'$}
            \end{prooftree}
            帰納法の仮定より,
            $\Gamma, y \mysncln{n_2} S' \myvdash{n_1} s \COL S$ ならば, $\Gamma, y \mysncln{n_2} S' \myvdash{n_2} [x \mapsto s] t' \COL T'$ である.
            \\
            $\Gamma, x \mysncln{n_1} S \myvdash{n_2} \mylam{y}{S'}{t'} \COL \Pi y \COL S'. T'$ 
            かつ $\Gamma \myvdash{n_1} s \COL S$, ならば,
            $\Gamma \myvdash{n_2} \mylam{y}{S'}{[x \mapsto s] t'} \COL \Pi y \COL S'. [x \mapsto s] T'$ を示したい.
            \\$\Gamma \myvdash{n_2} s \COL S$, のとき, \Cref{Weakening} より, $\Gamma, y \mysncln{n_2} S' \myvdash{n_1} s \COL S$.
            よって, 帰納法の仮定より, \(\Gamma, y \mysncln{n_2} S' \myvdash{n_2} [x \mapsto s] t' \COL [x \mapsto s] T'\).
            \\\fauxsc{T-Abs} によって, 目的の命題が得られる.
            \begin{prooftree}
                \AxiomC{$\Gamma \myvdash{n_2} S' \DCOL *$}
                \AxiomC{$\Gamma, y \mysncln{n_2} S' \myvdash{n_2} [x \mapsto s] t' \COL [x \mapsto s] T'$}
                \myplabel{T-Abs}
                \BinaryInfC{$\Gamma \myvdash{n_2} \mylam{y}{S'}{[x \mapsto s] t'} \COL \Pi x \COL S'. [x \mapsto s] T'$}
            \end{prooftree}
            \item{\fauxsc{T-App}} ($t = t_1 \mysp t_2, \quad T = T', \quad \Gamma; x \mysncln{n_1} S \myvdash{n_2} t_1 \mysp t_2 \COL T'$)\\
            帰納法の仮定より, $\Gamma \myvdash{n_1} s \COL S$, ならば, $\Gamma \myvdash{n_2} [x \mapsto s] t_1 \COL \Pi y \COL S' . T'$ かつ
            $\Gamma \myvdash{n_2} [x \mapsto s] t_2 \COL S'$.
            \\\fauxsc{T-App} より, 示したい命題が得られる.
            \begin{prooftree}
                \AxiomC{$\Gamma \myvdash{n_2} [x \mapsto s] t_1 \COL \Pi y \COL S' . T'$}
                \AxiomC{$\Gamma \myvdash{n_2} [x \mapsto s] t_2 \COL S'$}
                \myplabel{T-App}
                \BinaryInfC{$\Gamma \myvdash{n_2} [x \mapsto s] (t_1 \mysp t_2) \COL [x \mapsto s]T'$}
            \end{prooftree}
            \item{\fauxsc{T-PrevIntro}} ($t = \myprev{t_1}, \quad \Gamma; x \mysncln{n_1} S \myvdash{n_2} \myprev{t_1} \COL T$)\\
            $\Gamma \myvdash{n_2 - 1} t_1 \COL \bigcirc T$ が成立する.
            帰納法の仮定より, $\Gamma \myvdash{n_1} s \COL S$
            ならば, $\Gamma \myvdash{n_2 - 1} [x \mapsto s] t_1 \COL [x \mapsto s] \bigcirc T$.
            この判断を \fauxsc{T-PrevIntro} に適用する事によって, $\Gamma \myvdash{n_2} \myprev{[x \mapsto s] t_1} \COL [x \mapsto s] T$ が得られる.
            \item{\fauxsc{T-NextIntro}}\\
            \fauxsc{T-PrevIntro} の場合と同様.
        \end{inditem}
        % T-Conv
        % T-EqIntro
        % T-EqElim
        残りの場合は単純な帰納法の仮定の適用で示される.
    \end{proof}
\end{jlemma}
\vspace{10pt}

代入補題を用いて, 体系の型付けされた項は, 1ステップの評価によって
型が変わることがない保存と呼ばれる性質を \Cref{Preservation} に示す.

\begin{jtheorem}[保存]\label{Preservation}
    任意の $i  \in \mathbb{N}, n \in \mathbb{Z}$ に対して,
    $\Gamma \myvdash{n} t \COL T$ かつ $\myval{t}{d} \iarrow{i} \myval{t'}{d}$ ならば, 
    $\Gamma \myvdash{n} t'\COL T$ が成立する.
\begin{proof}
    \textbf{$\Gamma \myvdash{n} t\COL T$ の導出に関する構造帰納法による.}
    \begin{itemize}
        \item{\fauxsc{T-Var}} ($t = x$)\\
        $\myval{x}{d} \iarrow{i} \myval{t'}{d}$ であるような $t'$ は存在しない.
        \item{\fauxsc{T-Abs}} ($t = \mylam{x}{S}{t}$)\\
        $\myval{\mylam{x}{S}{t}}{d} \iarrow{i} t'$ であるような $t'$ は存在しない. 
        \item{\fauxsc{T-App}} ($t = t_1 \mysp t_2$)\\
        帰納法の仮定より, 以下の命題が与えられる. 
        \begin{equation*}\tag{IH1}
            (\Gamma \myvdash{n} t_1\COL \Pi x\COL S.T \land \myval{t_1}{d} \iarrow{i} \myval{t_1'}{d}) \Rightarrow \Gamma \myvdash{n} t_1'\COL \Pi x\COL S.T
        \end{equation*}
        \begin{equation*}\tag{IH2}
            (\Gamma \myvdash{n} t_2\COL S \land \myval{t_2}{d} \iarrow{i} \myval{t_2'}{d}) \Rightarrow \Gamma \myvdash{n} t_2'\COL S
        \end{equation*}
        \begin{enumerate}
            \item{\fauxsc{E-App1}}\\
            \begin{equation*}
                \myval{t_1 t_2}{d} \iarrow{i} \myval{t_1' t_2}{d}
            \end{equation*}
            $t_1' t_2$ の型導出は, \textrm{IH1} を用いて以下のように書ける.
            \begin{prooftree}
                \AxiomC{$\Gamma \myvdash{n} t_1'\COL \Pi x\COL S.T$}
                \AxiomC{$\Gamma \myvdash{n} t_2 \COL S$}
                \myplabel{T-App}
                \BinaryInfC{$\Gamma \myvdash{n} t_1' t_2 \COL T$}
            \end{prooftree}
            \item{\fauxsc{E-App2}}\\
            \begin{equation*}
                \myval{t_1 t_2}{d} \iarrow{i} \myval{t_1 t_2'}{d}
            \end{equation*}
            $v_1 t_2'$ の型導出は, \textrm{IH2} を用いて以下のようにかける.
            \begin{prooftree}
                \AxiomC{$\Gamma \myvdash{n} t_1\COL \Pi x\COL S.T$}
                \AxiomC{$\Gamma \myvdash{n} t_2' \COL S$}
                \myplabel{T-App}
                \BinaryInfC{$\Gamma \myvdash{n} t_1 t_2' \COL T$}
            \end{prooftree}
            \item{\fauxsc{E-Beta}} ($t_1 = \mylam{x}{S}{s_1}, \mysp t_2 = s_2, \mysp (\mylam{x}{S}{s_1}) s_2 \iarrow{i} [x \mapsto s_2] s_1$)\\
            $\Gamma \myvdash{n} (\mylam{x}{S}{s_1}) s_2$ の型導出により, 
            $\Gamma, x \mysncln{n} S \myvdash{n} s_1 \COL T'$ と $\Gamma \myvdash{n} s_2 \COL S$ が得られる.
            代入補題 (\Cref{Substitution}), を用いることで, $\Gamma \myvdash{n} [x \mapsto s_2] s_1 \COL [x \mapsto s_2] T'$, 
            が求まり, 示された.
        \end{enumerate}
        \item{\fauxsc{T-Conv}}\\
        容易に示される. 
        \item{\fauxsc{T-PrevIntro}} ($t = \myprev{s}$) \\
        帰納法の仮定より,  
            $\Gamma \myvdash{n-1} s \COL \bigcirc T$ かつ $\myval{t}{d} \iarrow{i} \myval{t'}{d}$ ならば, $\Gamma \myvdash{n-1} s'\COL T$.
        \begin{enumerate}
            \item {$\myprev{s} \iarrow{i} \myprev{s'}$}\\
            以下の導出により示される.
            \begin{prooftree}
                \AxiomC{$\Gamma \myvdash{n-1} s' \COL T$}
                \myplabel{T-PrevIntro}
                \UnaryInfC{$\Gamma \myvdash{n} \myprev{s'}\COL T$}
            \end{prooftree}
            \item{$\myprev{\mynext{s}} \iarrow{i} s$}\\
            $\Gamma \myvdash{n} \myprev{\mynext{s}} \COL T$ の型導出から, $\Gamma \myvdash{n} s \COL T$ が得られる. 
        \end{enumerate}
        \item{\fauxsc{T-NextIntro}} ($t = \mynext{s}$)\\
        \begin{enumerate}
            \item{$\mynext{s} \iarrow{i} \mynext{s'}$}\\
            上の場合と同様.
            \item{$s = \myprev{u}, \mynext{\myprev{u}} \iarrow{i} u, \quad T = \bigcirc T'$}\\
            $\Gamma \myvdash{n} t \COL T$ が以下のように導出できる.
            \begin{prooftree}
                \AxiomC{$\Gamma \myvdash{n} u \COL \bigcirc T'$}
                \myplabel{T-PrevIntro}
                \UnaryInfC{$\Gamma \myvdash{n + 1} \myprev{u} \COL T'$}
                \myplabel{T-NextIntro}
                \UnaryInfC{$\Gamma \myvdash{n} \mynext{\myprev{u}} \COL \bigcirc T'$}
            \end{prooftree}
            よって, $\Gamma \myvdash{n} u \COL \bigcirc T'$ が示される.
        \end{enumerate}
        % By the induction hypothesis, 
        %     If $\Gamma \myvdash{n+1} t\COL T$  and $\myval{t}{d} \iarrow{i} \myval{t'}{d}$ then $\Gamma \myvdash{n} t'\COL T$
        % \begin{enumerate}
        %     \item{$\mynext{t} \nrightarrow_{i}$}\\
        %     Immediate.
        %     \item{$\mynext{t} \iarrow{i} \mynext{t'}$}\\
        %     Type preservation holds by the following type derivation.
        %     \begin{prooftree}
        %         \AxiomC{$\Gamma \myvdash{n+1} t' \COL T$}
        %         \myplabel{T-NextIntro}
        %         \UnaryInfC{$\Gamma \myvdash{n} \mynext{t'} \COL \bigcirc T$}
        %     \end{prooftree}
        % \end{enumerate}
    \end{itemize}
\end{proof}
\end{jtheorem}
\vspace{10pt}

\begin{jtheorem}[進行]\label{Progress}
    $\bullet \myvdash{n} t\COL T$ であるとする. 任意の $i \in \mathbb{Z}$ に対して, 
    $t \in v^{i}$ であるか, $t \iarrow{i} t'$ であるような $t'$ が存在する.
    \begin{proof}
        \textbf{$\bullet \myvdash{n} t\COL T$ の導出に関する構造帰納法による.}
        \begin{itemize}
            \item{\fauxsc{T-Var}}\\
            環境が空であるので, この場合は可能でない.
            \item{\fauxsc{T-Abs}} ($t = \mylam{x}{T}{t_1}$)
            \begin{itemize}
                \item{$i \le 0$ である場合}\\
                定義より, $t \in v^{i}$.
                \item{$i > 0, \quad t_1 \in v^{i}$ である場合}\\
                定義より, $\mylam{x}{T}{t_1} \in v^{i}$
                \item{$i > 0, \quad t_1 \iarrow{i} t_1'$ である場合}\\
                \fauxsc{E-Abs} より, $\mylam{x}{T}{t_1} \iarrow{i} \mylam{x}{T}{t_1'}$.
            \end{itemize}
            \item{\fauxsc{T-App}}\\
            明らか.
            % 帰納法の仮定より, 以下の命題が得られる.
            % \begin{itemize}
            %     \item{\textrm{IH1}} $\myval{t_1}{d}$ is either a value or there exists $\myval{t_1}{d}$ s.t. $\myval{t_1}{d} \iarrow{i} \myval{t_1'}{d}$
            %     \item{\textrm{IH2}} $\myval{t_2}{d}$ is either a value or there exists $\myval{t_2}{d}$ s.t. $\myval{t_2}{d} \iarrow{i} \myval{t_2'}{d}$
            % \end{itemize}
            % \begin{enumerate}
            %     \item{$\myval{t_1}{d}$ と $\myval{t_2}{d}$ が値の場合}\\
            %     $\myval{t_1}{d}$ が値であるため, $\myval{t_1}{d}$ はラムダ抽象の形をしている.
            %     \fauxsc{E-Abs} より, $\myval{\mylam{x}{T}{t_{11}} t_2}{d} \iarrow{i} \myval{[x \mapsto t_2] t_{11}}{d}$ が得られる.
            %     \item{$\myval{t_1}{d}$ が値で, $\myval{t_2}{d}$ が値でない場合}\\
            %     \fauxsc{E-Abs} を適用して示すことができる.
            %     \item{$\myval{t_1}{d}$ が値でなく, $\myval{t_2}{d}$ が値の場合}\\
            %     帰納法の仮定より, $\myval{t_1}{d} \iarrow{i} \myval{t_1'}{d}$ となるような $\myval{t_1'}{d}$ が存在する. 
            %     \fauxsc{E-App} より, $\myval{t_1 t2}{d} \iarrow{i} \myval{t_1' t_2}{d}$ が得られる.
            %     \item{$\myval{t_1}{d}$ と $\myval{t_2}{d}$ がともに値出ない場合}\\
            %     3 の場合と同様. \fauxsc{E-App2} を用いて示される.
            % \end{enumerate}
            \item{\fauxsc{T-Conv}}\\
            帰納法の仮定より, 得られる.
            \item{\fauxsc{T-PrevIntro}} ($t = \myprev{t_1}$)
            \begin{itemize}
                \item{$i \le 1, \quad t_1 \iarrow{i-1} t_1'$ である場合}\\
                \fauxsc{E-Prev} より, $\myprev{t_1} \iarrow{i} \myprev{t_1}$
                \item{$i \le 1, \quad t_1 \in v^{i-1}$ である場合}\\
                $t_1$ はコード型であるため, $t_1 = \mynext{s}$ の形である.
                よって, \fauxsc{E-StageBeta} より, $\myprev{\mynext{s}} \iarrow{i} t_1$.
                \item{$i > 1, \quad t_1 \iarrow{i-1} t_1'$ である場合}\\
                \fauxsc{E-Prev} より, $\myprev{t_1} \iarrow{i} \myprev{t_1}$
                \item{$i > 1, \quad t_1 \in v^{i-1}$ である場合}\\
                定義より, $\myprev{t_1} \in v^{i}$.
            \end{itemize}
            \item{\fauxsc{T-NextIntro}} ($t = \mynext{t_1}$)
            \begin{itemize}
                \item{$t_1 \iarrow{i+1} t_1'$ である場合}\\
                \fauxsc{E-Next} より, $\mynext{t_1} \iarrow{i} t_1'$ である.
                \item{$t_1 \in v^{i+1}$ である場合}\\
                値の定義より, $t_1 \in v_{i+1}$ ならば, $\mynext{t_i} \in v_{i}$.
            \end{itemize}
        \end{itemize}
    \end{proof}
\end{jtheorem}
\vspace{10pt}

また, 項の正規形は一意に定まることを示すため, 強正規化性と合流性を示す.

まず, 強正規化性を示すために, \lamlf(Edinburgh logical framework) の強正規化性を仮定し, \lamlfcirc から \lamlf 
へ, 簡約と型付けを保存するような変換を\Cref{DefLamLFTransSN}に示す. そして背理法によって, \lamlfcirc が
強正規化性を持たないなら, \lamlf が強正規化性を持たないことを\Cref{SN}で示す.

\begin{jdefinition}[Edinburgh logical framework への変換]\label{DefLamLFTransSN}
    以下に\lamlfcirc の環境, カインド, 型, 項 を \lamlf のそれに対応させるような変換 $\sharp$ 及び, $\flat$ を定義する.
    定義されていない変換は恒等変換である. また, 型 \Unit \mysp は, 項 \unit \mysp からなるシングルトン型である. 
    \begin{itemize}
        \item $\strans{\mynext{t}} = \mylam{x}{\Unit}{\strans{t}}$
        \item $\strans{\myprev{t}} = \strans{t} \mysp \unit$
        \item $\strans{\myid{t}} = \mylam{x}{\Unit}{\strans{t}}$
        \item $\strans{\myidpeel{t_{eq}}{x}{t}} = (\mylam{a}{\ftrans{T}}{[x \mapsto a]}) (\strans{t_{eq}} \mysp \unit)$
    \end{itemize}
    \begin{itemize}
        \item $\ftrans{\bigcirc T} = \Pi x \COL \Unit. \ftrans{T}$
        \item ${\ftrans{\myeq{T}{a}{b}} = \Pi x \COL \Unit. \ftrans{T}}$
        \item $\ftrans{\Pi x \COL T . K} = \Pi x \COL \ftrans{T}. \ftrans{K}$
        \item ${\flat(\Gamma) = \{x \COL \flat(T) | (x \COL T) \in \Gamma\} \cup \{X \COL K | (X \COL K) \in \Gamma\}}$
    \end{itemize}
\end{jdefinition}
\vspace{10pt}

\begin{jtheorem}[強正規化性]\label{SN}
    $t$ が型付けされた項ならば, 無限の簡約列は存在しない.
    \begin{proof}
        \textbf{背理法による.}
        \lamlfcirc が強正規化性を持たないと仮定する. 
        このとき, \lamlfcirc で型付けされた項$t$が存在し, $t \rightarrow t_1 \rightarrow t_2 \rightarrow \cdots $
        のような無限長の簡約列が存在する. 
        \Cref{FiniteStageReduction}より, この簡約列には無限個の \mbeta-簡約 あるいは 同値型に関する簡約ステップが存在するから,
        よって, \Cref{BetaTransPreserve} と \Cref{EqualityTransPreserve} より, \lamlf でも $\strans{t}$ の無限長の簡約列が存在する.
        これは, \lamlf が強正規化性を持つことに矛盾する.
    \end{proof}
\end{jtheorem}
\vspace{10pt}

\FB

合流性は, 簡約関係の推移閉包となるような並行簡約を\Cref{FigParallelReduction}に定義し, 並行簡約
において, ダイアモンド性が満たされていることを\Cref{DiamondParallel}で示し, 証明する.

\FB

\begin{fig}{並列簡約の規則}
    \centering
    \footnotesize
    \infax[\fauxsc{EP-Var}]{
        x \parrow{i} x
    } \bcpnl
    \infrule[EP-Beta]{
        s \parrow{i} s'
        \andalso t \parrow{i} t'
        \andalso i = 0
    }{
        (\mylam{x}{T}{t})\mysp s \parrow{i} [x \mapsto s'] t'
    } \bcpnl
    \infrule[EP-Abs]{
        t \parrow{i} t'
    }{
        \mylam{x}{T}{t} \parrow{i} \mylam{x}{T}{t'}
    } \quad
    \infrule[EP-App]{
        t_1 \parrow{i} t_1'
        \andalso t_2 \parrow{i} t_2'
    }{
        t_1 \mysp t_2 \parrow{i} t_1' \mysp t_2'
    } \bcpnl
    \infrule[EP-Next]{
        t \parrow{i} t'
    }{
        \mynext{t} \parrow{i+1} \mynext{t'}
    } \quad
    \infrule[EP-Prev]{
        t \parrow{i-1} t'
    }{
        \myprev{t} \parrow{i} \myprev{t'}
    } \bcpnl
    \infrule[EP-StagedBeta]{
        t \parrow{i} t'
        \andalso i \le 1
    }{
        \myprev{\mynext{t}} \parrow{i} t'
    } \quad
    \infrule[EP-StagedEta]{
        t \parrow{i} t'
        \andalso i = 0
    }{
        \mynext{\myprev{t}} \parrow{i} t'
    } \bcpnl
    \infrule[EP-Equal]{
        t \parrow{i} t'
        \andalso i = 0
    }{
        \myidpeel{\myid{a}}{x}{t} \parrow{i} [x \mapsto a] t'
    }
    \label{FigParallelReduction}
\end{fig}

\FB

\begin{jlemma}[並行簡約のダイヤモンド性]\label{DiamondParallel}
    任意の自然数 $i$ と任意の項 $s$ について, $s \parrow{i} t_1$ かつ $s \parrow{i} t_2$ ならば, 
    $t_1 \parrow{i} t$ かつ $t_2 \parrow{i} t$ が成立する.
    \begin{proof}
        はじめに, 項 $M$ に対して, 項 $\tilde{M}$ を定義する.
        \begin{itemize}
            \item $M \equiv x$ ならば $\tilde{M} \equiv x$
            \item $M = \mylam{x}{T}{M_1}$ ならば $\tilde{M} = \mylam{x}{T}{\tilde{M_1}}$
            \item $M = M_1 M_2$ かつ $M$ \mbeta-基でないなら, $\tilde{M} = \tilde{M_1} \tilde{M_2}$
            \item $M = (\mylam{x}{T}{M_1}) M_2$ ならば $\tilde{M} = [x \mapsto \tilde{M_2}] \tilde{M_1}$
            \item $M = \mynext{M_1}$ かつ $M$ がステージ簡約基でないなら, $\tilde{M} = \mynext{\tilde{M_1}}$ 
            \item $M = \mynext{\myprev{M_1}}$ ならば $\tilde{M} = \tilde{M_1}$ 
            \item $M = \myprev{M_1}$ かつ $M$ がステージ簡約基でないなら $\tilde{M} = \myprev{\tilde{M_1}}$ 
            \item $M = \myprev{\mynext{M_1}}$ ならば $\tilde{M} = \tilde{M_1}$
            % \item If $M = \myidpeel{M_1}{x}{M_2}$ and $M$ is not a stage redex, $\tilde{M} = M$
            \item $M = \myidpeel{\myid{M_1}}{x}{M_2}$ ならば, $\tilde{M} = [x \mapsto M_1] \tilde{M_2}$
        \end{itemize}
        $M \parrow{i} N$ ならば, $N \parrow{i} \tilde{M}$ であることを示す. これは, 並行簡約のダイヤモンド性を示している.\\
        \textbf{$M$ の構造に関する帰納法.}
        \begin{itemize}
            \item{$M = x$}\\
            容易に示される.
            \item{$M = \mylam{x}{T}{M_1}, \quad \mylam{x}{T}{M_1} \parrow{i} \mylam{x}{T}{N}$}\\
            帰納法の仮定より, $N \parrow{i} \tilde{M_1}$ である.
            よって, $\mylam{x}{T}{N} \parrow{i} \mylam{x}{T}{\tilde{M_1}}$.
            \item{$M = M_1 M_2, \quad M_1 M_2 \parrow{i} N_1 N_2$ かつ $M$ が \mbeta-基出ない場合}\\
            帰納法の仮定より, $N_k \parrow{i} \tilde{M_k} (k = 1, 2)$ である. 
           これより, $N_1 N_2 \parrow{i} \tilde{M_1} \tilde{M_2}$ が示される.
            \item{$M = (\mylam{x}{T}{M_1}) M_2, \quad (\mylam{x}{T}{M_1}) M_2 \parrow{i} N$}
            \begin{itemize}
                \item{$N = [x \mapsto N_2] N_1$ である場合}\\
                帰納法の仮定より, $N_k \parrow{i} \tilde{M_k} (k = 1, 2)$ であり, よって ${N \parrow{i} [x \mapsto \tilde{M_2}] \tilde{M_1}}$ が成立する.
                \item{$N = N_0 N_2$}\\
                帰納法の仮定より, ${N_0 \parrow{i} \mylam{x}{T}{\tilde{M_1}}}$ かつ ${N_2 \parrow{i} \tilde{M_2}}$ が成り立つ.
                よって, $N_0 N_2 \parrow{i} [x \mapsto \tilde{M_2}] \tilde{M_1}$ が成立する.
            \end{itemize}
        \end{itemize}
        他のケースは, 単純な帰納法の仮定の適用により, 示される.
    \end{proof}
\end{jlemma}
\vspace{10pt}
\FB

\begin{jtheorem}[合流性]\label{ChurchRosser}
    $q \rightarrow^* r$ かつ $q \rightarrow^* s$ ならば, 
    $r \rightarrow^* t$ かつ $s \rightarrow^* t$ であるような項 $t$ が存在する.
    \begin{proof}
        \Cref{DiamondParallel} と \Cref{ParallelReductionTransClosure} より従う.
    \end{proof}
\end{jtheorem}
\vspace{10pt}

% next section: ./algorithmic.tex
