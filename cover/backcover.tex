\documentclass{tarticle}[10pt]
%%%%%%%%%%%%%%%%%%%%%%%%%%%%%%%%%%%%%%%%%%%%%%%%
%
% 卒論の背表紙を成する(複数人数対応) ver 1.1
%
%       2001/2/12  written by Takashi SUMIYOSHI
%
%%%%%%%%%%%%%%%%%%%%%%%%%%%%%%%%%%%%%%%%%%%%%%%%
\begin{document}
\kanjiskip=.3zw plus 3pt minus 3pt
\xkanjiskip=.3zw plus 3pt minus 3pt
\pagestyle{empty}
% 枠の余白の大きさ
\setlength{\fboxsep}{3mm}
% 平成何年度か
\def\thisyear{元}

\setlength{\textheight}{\paperheight}   % ひとまず紙面を本文領域に
\setlength{\topmargin}{-23mm}      % 上の余白を20mm(=1inch-5.4mm)に
\addtolength{\topmargin}{-\headheight}  %
\addtolength{\topmargin}{-\headsep}     % ヘッダの分だけ本文領域を移動させる
\addtolength{\textheight}{-10mm} 

%\setlength{\textwidth}{280mm}
\setlength{\textwidth}{\paperwidth}

% makebackcover
\def\makebackcover#1#2{\framebox[\textheight]{{\huge {\bf
#1}\hfill {\bf #2} \hspace{0.4cm} 令和 \rensuji*{\thisyear} 年度 }}}

\makebackcover
% {\normalsize \bf 依存型を備えた多段階計算の同値型に関する拡張}
{\Large \bf 依存型を備えた多段階計算の同値型に関する拡張}
{\Large 勝田 峻太朗}

%\makebackcover
%{卒論の背表紙を作成する Template File(複数人数対応)}
%{住吉 貴志}

\end{document}
